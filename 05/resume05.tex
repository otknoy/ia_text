\documentclass[11pt,a4paper]{jarticle}
\usepackage[dvipdfmx]{graphicx}
\usepackage{url}

\renewcommand{\baselinestretch}{1.05} 
\marginparwidth=0cm
\topmargin=-1cm
\headheight=0.3cm
\headsep=0.7cm
\oddsidemargin=0cm
\evensidemargin=0cm
%\textwidth=43zw
\textwidth=15.92cm
%\textheight=43.3\baselineskip
\baselineskip = 0.5744cm
\textheight=43\baselineskip

\itemsep=0.05\baselineskip
\parsep=0pt
\topsep=0.01\baselineskip
\partopsep=0pt
\listparindent=0zw

%% header and footer
\usepackage{fancyhdr}
\pagestyle{fancy}
\lhead{2014年度 春学期授業}
\chead{インタラクティブ・アート実習}
\rhead{担当教員: 松下 光範}
\cfoot{\thepage}
\renewcommand{\headrulewidth}{0pt}
\renewcommand{\footrulewidth}{0pt}

\usepackage{ascmac}
\usepackage{listings,jlisting}
\usepackage{color}
\definecolor{OliveGreen}{cmyk}{0.64,0,0.95,0.40}
\definecolor{colFunc}{rgb}{1,0.07,0.54}
\definecolor{CadetBlue}{cmyk}{0.62,0.57,0.23,0}
\definecolor{Brown}{cmyk}{0,0.81,1,0.60}
\definecolor{colID}{rgb}{0.63,0.44,0}
\definecolor{rulesepcolor}{gray}{0.666}
\lstset{
  language=Java,%プログラミング言語によって変える。
  basicstyle={\ttfamily\small},
  keywordstyle={\color{OliveGreen}},
  %[2][3]はプログラミング言語によってあったり、なかったり
  keywordstyle={[2]\color{colFunc}},
  keywordstyle={[3]\color{CadetBlue}},%
  commentstyle={\color{Brown}},
  %identifierstyle={\color{colID}},
  stringstyle=\color{blue},
  tabsize=2,
  %frame=trBL,
  %numbers=left,
  numberstyle={\ttfamily\small},
  breaklines=true,%折り返し
  %backgroundcolor={\color[gray]{.95}},
  framexleftmargin=0mm,
  frame=single,
  rulesepcolor=\color{rulesepcolor},
  captionpos=b
}


%%%%%%%%%%%%%%%%%%%%%%%%%%%%%%%%%%%%%%%%%%%%%%%%%%%%%%%%%%%%%%%%
\begin{document}
% title
\section*{\LARGE{第5講 様々なセンサを用いる}}
Arduino の Analog Input を用いて光センサや圧力センサ、曲げセンサの値を取得してみよう。

%%%%%%%%%%%%%%%%%%%%%%%%%%%%%%%%%%%%%%%%%%%%%%%%%%%%%%%%%%%%%%%%

\section{Processingの画面に文字を表示する}
Processing では、矩形や円のような図形だけではなく、文字を描くこともできます。
文字を表示するには text() を用います。
また、 PFont と textfont() を用いることによってフォントの種類や大きさを変更することができます。

Tools → Create Font の説明を!
分かり難いので図を入れる!

\begin{lstlisting}
 PFont font;

 void setup() {
   size(300, 300);

   // Create Font で作ったフォントを読み込む
   font = loadFont("Serif-48.vlw");
 }

 void draw() {
   background(255);

   textFont(font, 32); // 使うフォントとその大きさの指定
   fill(0);
   text("Interactive Art", 64, 64); // 文字を描く
 }
\end{lstlisting}


\section{Analog Input}
Analog Input の説明をほげほげ

Processing から Digital Input を使う場合はは arduino.digitalRead(pinNum); を使いましたが、
Analog Input を使うの場合は代わりに arduino.analogRead(pinNum); を使います。


\section{光センサを使う}
光センサとは、表面に当たる光の量に従って抵抗値が変化する電子部品です。
周りが暗いと抵抗値が大きく、明るいと抵抗値が小さくなります。

実習で使う光センサは CdSセル: CdS(硫酸カドミウム)を主成分とするものです。

\begin{itemize}
 \item \textbf{利点}
       \begin{itemize}
	\item 可視光線に対して高感度
	\item 小型で軽量
	\item 比較的安価
       \end{itemize}
 \item \textbf{欠点}
       \begin{itemize}
	\item 反応速度がやや遅め
	\item カドミウム = 有害物質 (ゼッタイに食うな!)
       \end{itemize}
\end{itemize}

% 光センサの写真でも入れる

\subsection*{センサの抵抗値を測ってみよう}
テスターを使ってほげほげ

\subsection*{光センサの値を取得する}
光センサの値を取得するためには、Analog Input を用います。
Analog Input を使う場合は A0 〜 A5 ピンを用いなければなりません。接続するピンに注意してください。

\subsubsection*{回路}
峻よろしく

\subsubsection*{プログラム}
\begin{lstlisting}
 import processing.serial.*;
 import cc.arduino.*;
 
 Arduino arduino;
 int sensorPin = 3; // 光センサを A3 番ピンに接続した場合
 
 void setup() {
   arduino = new Arduino(this, Arduino.list()[0], 57600);
   arduino.pinMode(sensorPin, Arduino.INPUT); // ピンモードを INPUT に
 }

 void draw() {
   int sensorValue = arduino.analogRead(sensorPin); // Analog Input
   background(255);

   fill(0);
   textFont(font, 24);
   text("sensor value: " + sensorValue, 50, 100);
}
\end{lstlisting}


\section{圧力センサを使う}
圧力センサとは、ほげほげ。

\section{曲げセンサを使う}
曲げセンサとは、ほげほげ。

前で見せますが高いのでおまえらにはやらねー。

説明だけ


\end{document}