\documentclass[11pt,a4paper]{jarticle}
\usepackage[dvipdfmx]{graphicx}
\usepackage{url}

\renewcommand{\baselinestretch}{1.05} 
\marginparwidth=0cm
\topmargin=-1cm
\headheight=0.3cm
\headsep=0.7cm
\oddsidemargin=0cm
\evensidemargin=0cm
%\textwidth=43zw
\textwidth=15.92cm
%\textheight=43.3\baselineskip
\baselineskip = 0.5744cm
\textheight=43\baselineskip

\itemsep=0.05\baselineskip
\parsep=0pt
\topsep=0.01\baselineskip
\partopsep=0pt
\listparindent=0zw

%% header and footer
\usepackage{fancyhdr}
\pagestyle{fancy}
\lhead{2014年度 春学期授業}
\chead{インタラクティブ・アート実習}
\rhead{担当教員: 松下 光範}
\cfoot{\thepage}
\renewcommand{\headrulewidth}{0pt}
\renewcommand{\footrulewidth}{0pt}

\usepackage{ascmac}
\usepackage{listings,jlisting}
\usepackage{color}
\definecolor{OliveGreen}{cmyk}{0.64,0,0.95,0.40}
\definecolor{colFunc}{rgb}{1,0.07,0.54}
\definecolor{CadetBlue}{cmyk}{0.62,0.57,0.23,0}
\definecolor{Brown}{cmyk}{0,0.81,1,0.60}
\definecolor{colID}{rgb}{0.63,0.44,0}
\definecolor{rulesepcolor}{gray}{0.666}
\lstset{
  language=Java,%プログラミング言語によって変える。
  basicstyle={\ttfamily\small},
  keywordstyle={\color{OliveGreen}},
  %[2][3]はプログラミング言語によってあったり、なかったり
  keywordstyle={[2]\color{colFunc}},
  keywordstyle={[3]\color{CadetBlue}},%
  commentstyle={\color{Brown}},
  %identifierstyle={\color{colID}},
  stringstyle=\color{blue},
  tabsize=2,
  %frame=trBL,
  %numbers=left,
  numberstyle={\ttfamily\small},
  breaklines=true,%折り返し
  %backgroundcolor={\color[gray]{.95}},
  framexleftmargin=0mm,
  frame=shadowbox,
  rulesepcolor=\color{rulesepcolor},
  captionpos=b
}


%%%%%%%%%%%%%%%%%%%%%%%%%%%%%%%%%%%%%%%%%%%%%%%%%%%%%%%%%%%%%%%%
\begin{document}

% title
\section*{\LARGE{第7講 モーターを制御する}}
Arduino のアナログ出力を使って、モーターの回転数を制御する。

%%%%%%%%%%%%%%%%%%%%%%%%%%%%%%%%%%%%%%%%%%%%%%%%%%%%%%%%%%%%%%%%

\section{モーターを制御するには}
モーターを制御する回路にはいくつか種類があるが、今回は配線が簡単な電界効果トランジスタ (FET: Field effect transistor) を使用した回路を用いる。
モーターを回転させるには、比較的大きな電流が必要なため、Arduino の出力では、モーターを回転させ続けるほどのパワーがない。
そのため、外部電源からモーターに電力を供給し、Arduino や Processing で制御するために必要となるのが FTE である。
FET を用いることで、モーターに流す電流を制御することができるので、モーターの回転数を変化させることができる。(電流が多く流れると早く回る。)

モーター (RE-140RA)

FET (2SK2232)

\section{回路を組み立てる}

配線図

今回はいつもよりも多くの電子部品を使います。それぞれ各自、以下の部品が手元にあるか、確認してください。
\begin{itemize}
 \item モーター(RE-140RA) 1個
 \item FET (2SK2232) 1個
 \item 赤外線距離センサ 1個
 \item 抵抗器(10kΩ) 1個
 \item 単4型乾電池 2個
 \item 電池ケース 1個
 \item ダイオード 1個
\end{itemize}

\subsection*{TRY1}
回路図を参考にして、モーター・FET と Arduino を接続してみる。
必ず、Arduino はPC から取り外した状態で配線すること。

\subsection*{TRY2}
Processing と Arduino を組み合わせて、モーターの回転数を制御する。
int val の数値を変化させて、モータの回転数の変化を確認する。( val の値は 0 ∼ 255 までの数値)

[ sample code ]
\begin{lstlisting}
 import processing.gainer.*;

 Gainer gainer;
   int val = 50;
   void setup(){
   gainer = new Gainer(this);
 }
 
 void draw(){
   gainer.analogOutput(0, val);
 }
\end{lstlisting}

\subsection*{TRY3}
Arduino に赤外線距離センサを接続し、取得した値でモーターの回転数を制御する。

[ Hint ]
 analogInputで取得した値を、analogOutputに出力できるようなプログラムを書く
 gainer.analogInput[0];

ain0 の値を取得する
 gainer.analogoutput(0, val);

aout0 に val を出力する

\end{document}