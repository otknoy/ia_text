\documentclass[11pt,a4paper]{jarticle}
\usepackage[dvipdfmx]{graphicx}
\usepackage{url}

\renewcommand{\baselinestretch}{1.05} 
\marginparwidth=0cm
\topmargin=-1cm
\headheight=0.3cm
\headsep=0.7cm
\oddsidemargin=0cm
\evensidemargin=0cm
%\textwidth=43zw
\textwidth=15.92cm
%\textheight=43.3\baselineskip
\baselineskip = 0.5744cm
\textheight=43\baselineskip

\itemsep=0.05\baselineskip
\parsep=0pt
\topsep=0.01\baselineskip
\partopsep=0pt
\listparindent=0zw

%% header and footer
\usepackage{fancyhdr}
\pagestyle{fancy}
\lhead{2014年度 春学期授業}
\chead{インタラクティブ・アート実習}
\rhead{担当教員: 松下 光範}
\cfoot{\thepage}
\renewcommand{\headrulewidth}{0pt}
\renewcommand{\footrulewidth}{0pt}

\usepackage{ascmac}
\usepackage{listings,jlisting}
\usepackage{color}
\definecolor{OliveGreen}{cmyk}{0.64,0,0.95,0.40}
\definecolor{colFunc}{rgb}{1,0.07,0.54}
\definecolor{CadetBlue}{cmyk}{0.62,0.57,0.23,0}
\definecolor{Brown}{cmyk}{0,0.81,1,0.60}
\definecolor{colID}{rgb}{0.63,0.44,0}
\definecolor{rulesepcolor}{gray}{0.666}
\lstset{
  language=Java,%プログラミング言語によって変える。
  basicstyle={\ttfamily\small},
  keywordstyle={\color{OliveGreen}},
  %[2][3]はプログラミング言語によってあったり、なかったり
  keywordstyle={[2]\color{colFunc}},
  keywordstyle={[3]\color{CadetBlue}},%
  commentstyle={\color{Brown}},
  %identifierstyle={\color{colID}},
  stringstyle=\color{blue},
  tabsize=2,
  %frame=trBL,
  %numbers=left,
  numberstyle={\ttfamily\small},
  breaklines=true,%折り返し
  %backgroundcolor={\color[gray]{.95}},
  framexleftmargin=0mm,
  frame=single,
  rulesepcolor=\color{rulesepcolor},
  captionpos=b
}


%%%%%%%%%%%%%%%%%%%%%%%%%%%%%%%%%%%%%%%%%%%%%%%%%%%%%%%%%%%%%%%%
\begin{document}
% title
\section*{\LARGE{第4講 明るさをはかる}}
Arduino の Analog Input を用いて光センサの値を取得する。

%%%%%%%%%%%%%%%%%%%%%%%%%%%%%%%%%%%%%%%%%%%%%%%%%%%%%%%%%%%%%%%%

\section{光センサ}
CdSセル: CdS(硫酸カドミウム)を主成分とする電子部品。
表面に当たる光の量に従って抵抗値が変化する。
周りが暗いと抵抗値が大きく、明るいと抵抗値が小さくなる。

\begin{itemize}
 \item \textbf{利点}
       \begin{itemize}
	\item 可視光線に対して高感度
	\item 小型で軽量
	\item 比較的安価
       \end{itemize}
 \item \textbf{欠点}
       \begin{itemize}
	\item 反応速度がやや遅め
	\item カドミウム = 有害物質
       \end{itemize}
\end{itemize}

\subsection*{光センサの値を取得する}
光センサの値を取得するためには、Analog Input を用います。
Analog Input を用いる場合は A0 〜 A5 ピンを用います。接続するピンに注意してください。

\subsubsection*{回路を組む}
峻よろしく

\subsubsection*{プログラムを書く}
Processing から Digital Input を使う場合はは arduino.digitalRead(pinNum); を使いましたが、
Analog Input を使うの場合は代わりに arduino.analogRead(pinNum); を使います。
\begin{lstlisting}
import processing.serial.*;
import cc.arduino.*;
 
Arduino arduino;
int opticalSensorPin = 3; // 光センサを A3 番ピンに接続した場合
 
void setup() {
  arduino = new Arduino(this, Arduino.list()[0], 57600);
  arduino.pinMode(opticalSensorPin, Arduino.INPUT); // ピンモードを INPUT に
}

void draw() {
  int sensorValue = arduino.analogRead(opticalSensorPin); // Analog Input
  println(sensorValue);
}
\end{lstlisting}


\section{Processingの画面に文字を表示する}
先ほどはコンソール(Processing IDE の下の黒いところ)にセンサから取得した値を表示しましたが、次は Processing の画面に表示してみましょう。

\subsubsection*{Processing で画面に文字を表示する}
Processing で文字を表示するためには text() を用います。
\begin{lstlisting}
 void setup() {
   size(400, 300);
 }

 void draw() {
   // x = 50, y = 100 のところから "Interactive Art" の文字列を描く
   text("Interactive Art", 50, 100);
 }
\end{lstlisting}

また、 PFont を用いることによってフォントの種類を変更することができます。

Tools → Create Font の説明を!
分かり難いので図を入れる!


\begin{lstlisting}
 PFont font;

 void setup() {
   size(400, 300);
 
   font = createFont("font_name", 32); // フォントの用意
   textFont(font, 24);
 }

 void draw() {
   text("Interactive Art", 50, 100);
 }
\end{lstlisting}

これらを利用して光センサから取得した値を画面に表示してみましょう。

\begin{lstlisting}
// Arduino版に書き換える
import processing.gainer.*;
Gainer myGainer;
PFont myFont;
void setup(){
	
size(???, ???);
	
colorMode(RGB, ???);
	
myGainer = new Gainer(this);
	
myFont = ??? (“Arial”, 24);
	
textFont(myFont);
	
textAlign( ??? );
	
myGainer.
???
;
}
void draw(){
	
background(?);
	
fill(???);
	
text(“value = ”+ myGainer.analogInput[0], ???, ???);
}
\end{lstlisting}


\begin{lstlisting}
 import processing.serial.*;
 import cc.arduino.*;

 Arduino arduino;
 int sensorPin = 6;
 Font font;
 
 void setup() {
   size(400, 300);

   arduino = new Arduino(this, Arduino.list()[0], 57600);
   arduino.pinMode(sensorPin, Arduino.INPUT);

   font = createFont("hogehoge_font", 24);
 }

 void draw() {
   int sensorValue = arduino.analogRead(sensorPin);

   background(255);
 
   textFont(font);
   textAlign(CENTER);
   fill(0);
   text("sensorValue = " + sensorValue, 100, 100);
 }

\end{lstlisting}
\end{document}

 