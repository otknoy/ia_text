\documentclass[11pt,a4paper]{jarticle}
\usepackage[dvipdfmx]{graphicx}
\usepackage{url}

\renewcommand{\baselinestretch}{1.05} 
\marginparwidth=0cm
\topmargin=-1cm
\headheight=0.3cm
\headsep=0.7cm
\oddsidemargin=0cm
\evensidemargin=0cm
%\textwidth=43zw
\textwidth=15.92cm
%\textheight=43.3\baselineskip
\baselineskip = 0.5744cm
\textheight=43\baselineskip

\itemsep=0.05\baselineskip
\parsep=0pt
\topsep=0.01\baselineskip
\partopsep=0pt
\listparindent=0zw

%% header and footer
\usepackage{fancyhdr}
\pagestyle{fancy}
\lhead{2014年度 春学期授業}
\chead{インタラクティブ・アート実習}
\rhead{担当教員: 松下 光範}
\cfoot{\thepage}
\renewcommand{\headrulewidth}{0pt}
\renewcommand{\footrulewidth}{0pt}

\usepackage{ascmac}
\usepackage{listings,jlisting}
\usepackage{color}
\definecolor{OliveGreen}{cmyk}{0.64,0,0.95,0.40}
\definecolor{colFunc}{rgb}{1,0.07,0.54}
\definecolor{CadetBlue}{cmyk}{0.62,0.57,0.23,0}
\definecolor{Brown}{cmyk}{0,0.81,1,0.60}
\definecolor{colID}{rgb}{0.63,0.44,0}
\definecolor{rulesepcolor}{gray}{0.666}
\lstset{
  language=Java,%プログラミング言語によって変える。
  basicstyle={\ttfamily\small},
  keywordstyle={\color{OliveGreen}},
  %[2][3]はプログラミング言語によってあったり、なかったり
  keywordstyle={[2]\color{colFunc}},
  keywordstyle={[3]\color{CadetBlue}},%
  commentstyle={\color{Brown}},
  %identifierstyle={\color{colID}},
  stringstyle=\color{blue},
  tabsize=2,
  %frame=trBL,
  %numbers=left,
  numberstyle={\ttfamily\small},
  breaklines=true,%折り返し
  %backgroundcolor={\color[gray]{.95}},
  framexleftmargin=0mm,
  frame=shadowbox,
  rulesepcolor=\color{rulesepcolor},
  captionpos=b
}


%%%%%%%%%%%%%%%%%%%%%%%%%%%%%%%%%%%%%%%%%%%%%%%%%%%%%%%%%%%%%%%%
\begin{document}

% title
\section*{\LARGE{第6講 フルカラーLEDを制御する}}
フルカラーLEDを制御し、任意の色で光らせる。
Gainer miniの「analog output」「digital output」の違いを試してみる。

%%%%%%%%%%%%%%%%%%%%%%%%%%%%%%%%%%%%%%%%%%%%%%%%%%%%%%%%%%%%%%%%


\subsection*{フルカラーLEDとは}
フルカラーLEDを用いると、光の三原色を一つのLEDの中で扱うことができる。
光の三原色を混色することで、様々の色の表現ができる。
白色に点灯させるには、すべての出力が均等になるようにする。
フルカラーLEDには、内部に赤、緑、青の3つのLEDが入っている。
外部に足が4本出ているものが多いが、これはアノード(+)またはカソード(-)が3つ分共通になっているためで、それぞれアノードコモン、カソードコモンと呼ぶ。
本実習では、カソードコモンのものを用いる。
足の長さで、それぞれが何に対応しているかが区別できるようになっている。

実習で用いるLEDは、それぞれの足が
 ・R:赤色
 ・K:­(カソード)
R
K
G  ・B:青色
 ・G:緑色
B に対応する。

\subsection*{フルカラーLEDの接続}
カソードコモンタイプのLEDは出力からGNDへ流れ出す形で制御する。
本実習で用いるカソードコモンタイプの場合、赤・緑・青・カソードの4本の足がある。
Gainer miniの出力から、それぞれの色に適した値の抵抗器を経由して接続し、カソードはGNDに接続する。


回路図 その1
回路図 その2


\subsection*{パターンによって動作を振り分ける}
条件分岐には通常 if 文を使うが、数多くの振り分けを行うとプログラムが頻雑になってしまう。
こういった場合の時には、switch 文を使うと便利である。

switch文では、if文のように true と false の2択ではなく、変数の内容に応じて適切な場所にワープさせることが可能である。

\begin{lstlisting}
 switch(<<入力変数>>){
   case <<パターン1>> : ←コロン
     <<入力変数 == パターン1の時の処理>>
     break; ← セミコロン
   case<<パターン2>> :
     <<入力変数 == パターン2の時の処理>>
     break;
   default :
     <<どのcaseにも当てはまらなかった時の処理>>
     break;
 }
\end{lstlisting}

\subsection*{TRY1}
Processingで switch 文を用いて、キーボードの 1 が押されたら赤、2 が押されたら緑、3 が押されたら青、4 が押されたら白の四角形が表示され、プログラムの起動時や、他のキーが押されれば黒の四角形が表示されるようなプログラムを作成する。

[ Hint 1 ]
 キーボードの押したキーの判別は システム変数のkeyを用いる ことで簡単に取得できる。
[ Hint 2 ]
 case以下でのキーの指定は、
 ” ”(ダブルクォーテーション)ではなく、’’(シングルクォーテーション)で囲む
[ Hint 3 ]
 四角形の色の切り替えには fill(); を用いる
[ Hint 4 ]
...(前略)...
\begin{lstlisting}
 void draw(){
   switch(key){
     case ‘1’:
       fill(???, ???, ???);
       break;
 
      ...(中略)...
   }
   rect(???, ???, ???, ???);
 }
\end{lstlisting}

\section{Gainer mini の Digital Output を用いる}
Gainer mini の doutピンを用いて、電流のオン・オフ制御を簡単に切り替えることができる。

setHigh(int ch);
chで指定したdoutピンの状態をHighにする(ONにする)
setLow(int ch);
chで指定したdouピンの状態をLowにする(OFFにする)

\subsection*{TRY2}
1ページ目の「回路図 その1」を参考にして、デジタル出力を使った回路を組み立てる。(抵抗は、赤には330Ω、緑・青には180Ωを用いる。)

\subsection*{TRY3}
TRY1 のプログラムを改造して、画面に表示される四角形の色と、フルカラーLEDの色が同じになるようなプログラムを作成する。
黒の時はLEDを消灯する。

[ Hint1 ]
(回路図通り組み立てていれば)赤色はdout0、緑色はdout1、青色はdout2にそれぞれ対応する。

[ Hint2 ]
\begin{lstlisting}
 import processing.gainer.*;
 Gainer gainer;
 void setup() {
   size(200, 200);
   colorMode(RGB, 256);
   rectMode(CENTER);
   gainer = new Gainer(this);
 }
 
 void draw() {
   switch(key) {
     case '1':
       fill(255, 0, 0);
       gainer.setHigh(0);
       gainer.setLow(1);
       gainer.setLow(2);
       break;
     case '2':
       <<緑色にする時の処理>>
     case '3':
       <<青色にする時の処理>>
     case '4':
       <<白色にする時の処理>>
     default:
       <<LEDを消す時の処理>>
   }
   rect(width/2, height/2, 100, 100); 
 }
\end{lstlisting}

\section{Gainer mini の Analog Output を用いる}
Gainer mini の aout ピンを用いることで、アナログ値を出力することができる。
アナログ出力を用いることで、フルカラーLEDで様々な色を表現することができる。

analogOutput(int ch, int value);
ch で指定したアナログポート value で指定した値を出力する。
( value は 0∼255 までの 256階調 )

余談:アナログ出力の正体
 Gainer mini でアナログ出力と呼んでいるものは、実際には電圧が変化するものではなく、PWM (Pulse Width
Modulation: パルス幅変調)と呼ばれる方式で擬似的に実現している。
 例えば、値が1に設定した場合、出力信号は非常に短い区間だけ5Vで、残りの区間は0Vになる。値を127に設定した
場合は、半分の区間が5Vで残りの半分は0Vになっている。値を255に設定したときは常に5Vが出力される。

\subsection*{TRY4}
1 ページ目の「回路図 その2」を参考にして、TRY2 で組み立てた回路を改造し、アナログ出力を使った回路を組み立てる。

\subsection*{TRY5}
Processing でマウスの動きに反応して、フルカラー LED の色を制御できるようなプログラムを作成する。
マウスは 2 軸の値しか取れないので、RGB のいずれかの値を固定値にする。

[ Hint 1 ]
 マウスのX軸の値は mouseX で、マウスの Y 軸の値は mouseY で取得できる。
 
[ Hint 2 ]
\begin{lstlisting}
 import processing.gainer.*;
 Gainer gainer;
 void setup() {
   size(255, 255);
   colorMode(RGB, 256);
   gainer = new Gainer(this);
 }
 
 void draw() {
   background(???, ???, ???);
   gainer.analogOutput(0, ???);
   gainer.analogOutput(1, ???);
   gainer.analogOutput(2, ???);
 }
\end{lstlisting}

\end{document}