\documentclass[11pt,a4paper]{jarticle}
\usepackage[dvipdfmx]{graphicx}
\usepackage{url}

\renewcommand{\baselinestretch}{1.05} 
\marginparwidth=0cm
\topmargin=-1cm
\headheight=0.3cm
\headsep=0.7cm
\oddsidemargin=0cm
\evensidemargin=0cm
%\textwidth=43zw
\textwidth=15.92cm
%\textheight=43.3\baselineskip
\baselineskip = 0.5744cm
\textheight=43\baselineskip

\itemsep=0.05\baselineskip
\parsep=0pt
\topsep=0.01\baselineskip
\partopsep=0pt
\listparindent=0zw

%% header and footer
\usepackage{fancyhdr}
\pagestyle{fancy}
\lhead{2014年度 春学期授業}
\chead{インタラクティブ・アート実習}
\rhead{担当教員: 松下 光範}
\cfoot{\thepage}
\renewcommand{\headrulewidth}{0pt}
\renewcommand{\footrulewidth}{0pt}

\usepackage{ascmac}
\usepackage{listings,jlisting}
\usepackage{color}
\definecolor{OliveGreen}{cmyk}{0.64,0,0.95,0.40}
\definecolor{colFunc}{rgb}{1,0.07,0.54}
\definecolor{CadetBlue}{cmyk}{0.62,0.57,0.23,0}
\definecolor{Brown}{cmyk}{0,0.81,1,0.60}
\definecolor{colID}{rgb}{0.63,0.44,0}
\definecolor{rulesepcolor}{gray}{0.666}
\lstset{
  language=Java,%プログラミング言語によって変える。
  basicstyle={\ttfamily\small},
  keywordstyle={\color{OliveGreen}},
  %[2][3]はプログラミング言語によってあったり、なかったり
  keywordstyle={[2]\color{colFunc}},
  keywordstyle={[3]\color{CadetBlue}},%
  commentstyle={\color{Brown}},
  %identifierstyle={\color{colID}},
  stringstyle=\color{blue},
  tabsize=2,
  %frame=trBL,
  %numbers=left,
  numberstyle={\ttfamily\small},
  breaklines=true,%折り返し
  %backgroundcolor={\color[gray]{.95}},
  framexleftmargin=0mm,
  frame=single,
  rulesepcolor=\color{rulesepcolor},
  captionpos=b
}


%%%%%%%%%%%%%%%%%%%%%%%%%%%%%%%%%%%%%%%%%%%%%%%%%%%%%%%%%%%%%%%%
\begin{document}

% title
\section*{\LARGE{第3講 明るさをはかる}}
光センサと Arduino を用いて明るさの変化を計測する。

%%%%%%%%%%%%%%%%%%%%%%%%%%%%%%%%%%%%%%%%%%%%%%%%%%%%%%%%%%%%%%%%

\section{光センサとは}
CdSセル: CdS(硫酸カドミウム)を主成分とする電子部品。
表面に当たる光の量に従って抵抗値が変化する。
周りが暗いと抵抗値が大きく、明るいと抵抗値が小さくなる。

\subsection*{利点}
\begin{itemize}
 \item 可視光線に対して高感度
 \item 小型で軽量
 \item 比較的安価
\end{itemize}
	    
\subsection*{欠点}
\begin{itemize}
 \item 反応速度がやや遅め
 \item カドミウム=有害物質
\end{itemize}


\section{回路を組み立てる}

回路図

\subsection*{TRY1}
回路図を参考にして、光センサと Arduino を接続してみる。
その際必ず、Arduino は PC から取り外した状態で配線すること!

\subsubsection*{Processing から Arduino の アナログポートの値を取得する}

\subsection*{TRY2}
TRY1 で組み立てた光センサの値を、println を使って、ログに表示してみる。

[Hint 1]
println (表示したい値);

[Hint 2]

\begin{lstlisting}
 import processing.gainer.*;
 Gainer gainer;

 void setup(){
   gainer = new Gainer (this);
   gainer.???;
 }

 void draw(){
   ???(???);
 }
\end{lstlisting}


\section{Processingで文字を表示する}

PFont myFont;
PFont オブジェクトの宣言。
プログラムの頭に書く。
複数宣言することで、違う種類のフォントを使用することができる。

 myFont : PFont オブジェクトの名前
createFont(“useFont”, 32);
 フォントを作成し、PFontオブジェクトに読み込む命令。void setup の中に書く。
  “useFont” : 使用したいフォント名 (Arial, MS Gothic, etc...)
                  ※IDE上のタブのTools → Create Font... から 確認可能
  
数字
: 作成するフォントのサイズ (大きくしすぎると、処理が重くなる。)
textFont(myFont, 24);
 使用するフォントの選択をする命令。
  myFont = 使用するPFontオブジェクトの名前
  数字 = フォントサイズの指定
textAlign(LEFT);
 文字の揃え位置を指定する命令。
  LEFT
: 左揃え(初期設定、宣言しなかった場合も)
  CENTER : 中央揃え
  RIGHT
: 右揃え
text(“hoge”, x, y);
 
 文字を書く命令。
  “hoge” : 表示文字列
  x!! : x座標の位置
  y!! : y座標の位置

\subsection*{TRY3}
前回の資料も参考にしながら、Processing で 文字列を書いてみる。

[tips]: ウインドウ画面の幅や高さは、それぞれ width, height 変数で取得することができる。


\subsection*{TRY4}
TRY2、TRY3を参考にし、Processing の文字列 (text) で光センサの値を表示する。

[Hint]
\begin{lstlisting}
import processing.gainer.*;
Gainer myGainer;
PFont myFont;
void setup(){
	
size(???, ???);
	
colorMode(RGB, ???);
	
myGainer = new Gainer(this);
	
myFont = ??? (“Arial”, 24);
	
textFont(myFont);
	
textAlign( ??? );
	
myGainer.
???
;
}
void draw(){
	
background(?);
	
fill(???);
	
text(“value = ”+ myGainer.analogInput[0], ???, ???);
}
 
\end{lstlisting}

\end{document}